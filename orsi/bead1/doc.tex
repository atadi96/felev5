\documentclass[12pt]{article}

%margó méretek
\usepackage[a4paper,
inner = 25mm,
outer = 25mm,
top = 25mm,
bottom = 25mm]{geometry}


%packagek, ha használni akarunk valamit menet közben
\usepackage{lmodern}
\usepackage[magyar]{babel}
\usepackage[utf8]{inputenc}
\usepackage[T1]{fontenc}
\usepackage[hidelinks]{hyperref}
\usepackage{graphicx}
\usepackage{amssymb}
\usepackage{epstopdf}
\usepackage{setspace}
\usepackage[nottoc,numbib]{tocbibind}
\usepackage{setspace}
\usepackage{spverbatim}

\setstretch{1.2}
\begin{document}
	
	

% a címlap, túl sokat nem kell módosítani rajta, leszámítva a neved/neptunodat (dátumot)
\begin{titlepage}
	\vspace*{0cm}
	\centering
	\begin{tabular}{cp{1cm}c}
		\begin{minipage}{4cm}
			\vspace{0pt}
			\includegraphics[width=1\textwidth]{elte_cimer}
		\end{minipage} & &
		\begin{minipage}{7cm}
			\vspace{0pt}Eötvös Loránd Tudományegyetem \vspace{10pt} \newline
			Informatikai Kar \vspace{10pt} \newline
			Programozási Nyelvek és Fordítóprogramok Tanszék
		\end{minipage}
	\end{tabular}
	
	\vspace*{0.2cm}
	\rule{\textwidth}{1pt}
	
	\vspace*{3cm}
	{\Huge Osztott rendszerek specifikációja és implementációja }
	
	\vspace*{0.5cm}
	{\normalsize IP-08bORSIG}
	
	\vspace{2cm}
	{\huge Dokumentáció az 1. beadandóhoz}
	
	\vspace*{5cm}
	
	{\large \verb|Abonyi-Tóth Ádám| } % (név)
	
	{\large \verb|DKC31P| }  % (neptun)
		
	
	\vfill
	
	\vspace*{1cm}
	2017. október 26. % (dátum)
\end{titlepage}

\section{Kitűzött feladat}

Adott egy bemeneti fájl, amelynek sorait szavanként hash-elni kell.
Egy szó hashértéke a benne lévő karakterek hashértékének összege.
A szó egy \verb|'c'| betűjének hashkódja (\verb|'kód'|) az alábbi módon áll elő:
\\
\verb|  'kód' : Nemnegatív egész szám := 1638; (0x666)| \\
\verb|  ha 'c' ASCII értéke páratlan:| \\
\verb|      a 'kód'-ot bitenként shifteljük balra 11-et| \\
\verb|  egyébként:| \\
\verb|      'kód'-ot shifteljük 6-ot bitenként balra.| \\
\verb|  'kód' := 'kód' XOR ('c' ASCII értéke BITENKÉNTI ÉSELVE 255-el); (0xFF)| \\
\verb|  Ha az így kapott 'kód' prím szám, akkor bitenként 'vagy'-oljuk, amennyiben| \\
\verb|  nem prím, 'és'-eljük össze 305419896-tal. (0x12345678)|
\\
A program feladata az, hogy a bemeneti fájl (\verb|input.txt|) sorait párhuzamosan hash-elje,
és az eredményt kiírja (a sorok sorrendjét megtartva) a kimeneti fájlba (\verb|output.txt|).
\\
A bemeneti fájl első sora egy $N$ természetes számot tartalmaz, utána pedig $N$ sornyi
szöveges információ található. Feltehetjük, hogy a szövegben az angol ábécé betűit
használó szavak, valamint általános írásjelek (pont, vessző, kérdő- és felkiáltójel,
aposztróf, kettőspont stb.) találhatóak.
\\
Példa bemenet: \\
\verb|4| \\
\verb|Never gonna give you up, never gonna let you down| \\
\verb|Never gonna run around and desert you| \\
\verb|Never gonna make you cry, never gonna say goodbye| \\
\verb|Never gonna tell a lie and hurt you| \\

\section{Felhasználói dokumentáció}

\subsection{Rendszer-követelmények, telepítés}

A programunk több platformon is futtatható, dinamikus függősége nincsen, bármelyik, manapság használt PC-n működik. Külön telepíteni nem szükséges, elég a futtatható állományt elhelyezni a számítógépen.

\subsection{A program használata}

A program használata egyszerű, külön paraméterek nem kötelezőek, így intézőből
is indítható. Ha a programot paraméterek nélkül futtatjuk, a bemenetét a futtatható
állomány mellett kell elhelyezni az \textit{input.txt} nevű fájlban. Ha pontosan egy
parancssori paramétert kap a program, akkor az így megadott paraméter lesz a bemeneti
fájl elérési útja.
\\
Figyeljünk az inputfájlban található adatok helyességére és megfelelő tagolására, mivel
az alkalmazás külön ellenőrzést nem végez erre vonatkozóan. A futás során az alkalmazás
mellett található \textit{output.txt} fájl tartalmazza a kapott eredményt, ahol az
\textit{i}-ik sor a bemeneti fájl \textit{i+1}-ik sorának hashértéke.

\section{Fejlesztői dokumentáció}

\subsection{Megoldási mód}

A kódunkat logikailag két részre bonthatjuk, egy fő-, illetve több alfolyamatra.
A fő folyamatunkat a \verb|main()| és a \verb|read()| függvények fogják megvalósítani,
melyek feladata a bementi fájl beolvasása, illetve az alfolyamatok munkavégzési
idejének mérése. Az alfolyamatok a bemeneti állomány egy-egy sorának hash-eléséért
felelősek. A végeredményt a főfolyamat írja ki a kimeneti állományba.

\subsection{Implementáció}

A bemeneti fájlt \verb|std::vector<std::string>| típussal fogjuk megvalósítani,
míg az alfolyamatokat egy \verb|std::future<std::string>| típusparaméterű vektorban
fogjuk tárolni. A szükséges $N$ folyamatot az \verb|std::async()| függvény
segítségével, azonnal fogjuk új szálon indítani, paraméterül a végrehajtandó
\verb|process_line()| függvényt, illetve a szöveg egy sorát fogjuk átadni.

A \verb|process_line()| függvény feladata egy sor szavainak hash-elése, és a kiszámolt
hash-értékekből a kimenet sorainak összeállítása. Az implementáció használhatott volna
\verb|std::istream_iterator<T>|-on alapuló \verb|std::transform()| függvényt,
de az iterátor csak pointeren keresztül érte volna el a stream-eket, ez pedig a
feladat egyszerűségéhez képest felesleges overhead-et jelentett volna.

A szavak hashértékének kiszámítása triviális, a karakterek hashértékének számítása
szigorúan a feladat specifikációja alapján történik. Az \verb|is_prime(uint32_t)|
függvényben azért van kibontva az első 5 prímmel való oszthatóság, mert a modern
fordítók alkalmazhatnak olyan optimalizációkat, amik a fordítási időben ismert
értékekkel való maradékszámítást gyorsabbá teszik, mint egy változó esetében.
A \verb|nontrivial_prime(uint32_t)| függvény egy template paraméterrel rendelkezik,
mely azt jelzi, hogy mettől kezdje vizsgálni a prímszámokat. Mivel ez már fordítási
időben ismert, érdemesebb így, template paraméterként átadni, hiszen így teret
adhatunk egyéb, fordítási idejű optimalizációknak.

A feladat egyszerűsége révén egyetlen forrásfájlban, a \verb|main.cpp|-ben található a teljes implementációs kód.

\subsection{Fordítás menete}

A programunk forráskódját a \verb|main.cpp| fájl tartalmazza. A fordításhoz
elengedhetetlen egy \verb|C++11| szabványt támogató fordítóprogram a rendszeren.
Ehhez használhatjuk az \textit{MSVC}, \textit{g++} és \textit{clang} bármelyikét,
de \textit{Windows} operációs rendszer alatt \textit{MinGW} fordítóval a fordítás sikertelen lehet,
mert ezen a platformon az \verb|std::future| típus implementációja hiányozhat.

A fordítás menete (4.8.4-es verziójú g++ használata esetén) a következő:  \verb|g++ main.cpp|
\verb|-std=c++11 -pthread|.
A speciális, \verb|-std=c++11| kapcsoló azért
szükséges, mert alapértelmezés szerint ez a verziójú fordítóprogram még a régi, C++98-as
szabványt követi, melyben a felhasznált nyelvi elemek még nem voltak jelen.
A \verb|-pthread| kapcsolóra azért van szükség, hogy a fordító, illetve az operációs
rendszer megengedje a programnak a többszálúság használatát.

\subsection{Tesztelés}

A program tesztelése két részből állt, az egyik a megoldás helyességét vizsgálta, a másik
a kód sebességét.

A helyesség vizsgálatához a programot lefuttattam a feladathoz mellékelt bemeneti fájlokra,
és \verb|git diff --no-index --color-words| paranccsal ellenőriztem, hogy nincs eltérés az
elvárt és a kapott eredmények között.

A sebesség vizsgálatához a ugyanazt a szöveget tördeltem több sorra, ezzel
ellenőrizve, hogy hogyan befolyásolja a futásidőt az állandó mennyiségű feladatra
kiosztott szálak száma. A teszteket futtató processzor 4 logikai magot működtetett
2.9GHz-es órajelen. Az eredmények alább láthatóak.\\[12px]
\resizebox{\textwidth}{!}{
	\begin{tabular}{l|*{14}{c}}
		Szálak száma & 1 & 2 & 3 & 4 & 5 & 6 & 7 & 8 & 9 & 10 & 11 & 12 & \dots & 16\\
		\hline
		Futásidő (s) & 1.082 & 0.551 & 0.434 & 0.349 & 0.449 & 0.328 & 0.312 & 0.305 & 0.302 & 0.316 & 0.316 & 0.318 & \dots & 0.316\\
		\hline
		Összköltség  & 1.082 & 1.102 & 1.302 & 1.396 & 2.246 & 1.969 & 2.181 & 2.443 & 2.717 & 3.164 & 3.477 & 3.815 & \dots & 5.06
	\end{tabular}
}\\[12px]

A táblázat első négy oszlopában az összköltség \textit{(futásidő $\times$ szálak száma)}
nagyjából állandó, tehát itt a futásidő megközelítőleg fordítottan arányos 
a processzek számával.

A további oszlopokból az derül ki, hogy az összköltség hirtelen megnő, amint az
indított szálak mennyisége meghaladja a processzor logikai magjainak számát,
ami azt jelenti, hogy onnantól kezdve a gyorsulás nem lesz olyan nagy mértékű,
mint amíg nem használtunk ki minden processzormagot. Azt is láthatjuk, hogy a
futásidő egy idő után stabilizálódik $0.316s$ körül, akárhány szálat indítunk.


\end{document}