\documentclass[12pt]{article}

%margó méretek
\usepackage[a4paper,
inner = 25mm,
outer = 25mm,
top = 25mm,
bottom = 25mm]{geometry}


%packagek, ha használni akarunk valamit menet közben
\usepackage{lmodern}
\usepackage[magyar]{babel}
\usepackage[utf8]{inputenc}
\usepackage[T1]{fontenc}
\usepackage[hidelinks]{hyperref}
\usepackage{graphicx}
\usepackage{amssymb}
\usepackage{epstopdf}
\usepackage{setspace}
\usepackage[nottoc,numbib]{tocbibind}
\usepackage{setspace}
\usepackage{spverbatim}

\setstretch{1.2}
\begin{document}
	
	

% a címlap, túl sokat nem kell módosítani rajta, leszámítva a neved/neptunodat (dátumot)
\begin{titlepage}
	\vspace*{0cm}
	\centering
	\begin{tabular}{cp{1cm}c}
		\begin{minipage}{4cm}
			\vspace{0pt}
			\includegraphics[width=1\textwidth]{elte_cimer}
		\end{minipage} & &
		\begin{minipage}{7cm}
			\vspace{0pt}Eötvös Loránd Tudományegyetem \vspace{10pt} \newline
			Informatikai Kar \vspace{10pt} \newline
			Programozási Nyelvek és Fordítóprogramok Tanszék
		\end{minipage}
	\end{tabular}
	
	\vspace*{0.2cm}
	\rule{\textwidth}{1pt}
	
	\vspace*{3cm}
	{\Huge Osztott rendszerek specifikációja és implementációja }
	
	\vspace*{0.5cm}
	{\normalsize IP-08bORSIG}
	
	\vspace{2cm}
	{\huge Dokumentáció az 2. beadandóhoz}
	
	\vspace*{5cm}
	
	{\large \verb|Abonyi-Tóth Ádám| } % (név)
	
	{\large \verb|DKC31P| }  % (neptun)
		
	
	\vfill
	
	\vspace*{1cm}
	2017.november 30. % (dátum)
\end{titlepage}

\section{Kitűzött feladat}

Feladatunk annak eldöntése, hogy egy adott halmaznak létezik e olyan részhalmaza, melyben található elemek összege pontosan megegyezik egy előre megadott számmal!
\\
A szükséges adatokat a program 3 parancssori paraméteren keresztül kapja.
\\
Az első, egy egész értékű adat, mely a feladatban definiált számot jelzi, ezt kell valamilyen módon elérni a halmazelemek összegével.
\\
A második, egy fájl neve - ez tartalmazza a kiinduló halmaz elemeit (inputfájl). A felépítése az alábbi: A fájl első sorában egy nemnegatív egész szám (N) áll - a kiinduló halmazunk elemszáma tehát N. A következő sorban összesen N db egész számot olvashatunk (pozitív és negatív egyaránt), melyek a halmaz elemeit jelölik (sorrendiséget nem kötünk meg köztük).
\\
Egy megfelelő bemeneti fájl (például data.txt) ekkor:
\\
\verb|6| \\
\verb|3 34 4 17 5 2|
\\
A harmadik paraméter, annak a fájlnak a neve, melybe a feladat megoldása során kapott választ kell írni.
\\
A kimeneti fájlban található információ az alábbi legyen:
\\
\verb|   létezik N összegű részhalmaz:| \\
\verb|      'May the subset be with You!'| \\
\verb|   nem létezik ilyen részhalmaz:| \\
\verb|'I find your lack of subset disturbing!'| \\

\section{Felhasználói dokumentáció}

\subsection{Rendszer-követelmények, telepítés}

A program futtatásához szükséges a PVM rendszer megléte. Telepítéséhez a két bináris
állomány, a \textit{master} és a \textit{child} megfelelő helyen való elhelyezése szükséges.

\subsection{A program használata}

A program futtatásához előbb lépjünk be PVM-be, és inicializáljuk tetszés szerint a blade-eket.
Az indítás a PVM-en belül a következő paranccsal történik:
\\
\verb|spawn -> master sum input.txt output.txt|,
\\
ahol \verb|sum| a kérdéses összeg, \verb|input.txt| a bemeneti fájl neve, \verb|output.txt| pedig
a kimeneti fájl neve a felhasználó könyvtárához képest.
\\
Figyeljünk az inputfájlban található adatok helyességére és megfelelő tagolására, mivel
az alkalmazás külön ellenőrzést nem végez erre vonatkozóan. Futás után a paraméterben megadott
fájl tartalmazza a kapott eredményt.

\section{Fejlesztői dokumentáció}

\subsection{Megoldási mód}

A kódot logikailag három egységre osztjuk: a probléma reprezentációja, a főprogram, és a megoldó
program. A főprogram feladata a be- és kimenet kezelése, a gyermeké a feladat elosztott megoldása,
a harmadik pedig a probléma (és alproblémák) konzisztenciáját biztosítja.
A végeredményt a főfolyamat írja ki a kimeneti állományba.

\subsection{Implementáció}

A probléma reprezentációja egész számok egy sorozata, és a kívánt összeg. Ehhez a kód a \verb|problem.hpp|
fájlban található. Az implementációs osztály tartalmazza a domain-hez tartozó műveleteket is:
alproblémák elkészítése, a teljes probléma küldése és fogadása PVM-mel, illetve a probléma állapotának
felmérése (megoldott, megoldhatatlan, nem eldönthető).
\\
A \verb|master.cpp| fájlban van a főprogram megvalósítása. A \verb|main| függvény létrehozza a kezdeti
feladatot az összeggel és a bemeneti fájlban lévő számokkal, majd a PVM segítségével meghív egy gyermek
folyamatot a feladat megoldására. Amint a gyermek processz végzett, a program kiírja az eredményt a
harmadik paraméterben kapott kimeneti fájlba.
\\
A gyermek folyamat először fogadja a szülőtől a megoldandó problémát. Ha a probléma állapota ismert
(megoldott, vagy megoldhatatlan), akkor visszaadja ezt az értéket, ha nem akkor a gyermek 
\textit{Divide and Conquer} módszert alkalmazva próbálja megszerezni az eredmény. Ehhez a problémát
felosztja két részproblémára, majd megpróbál létrehozni egy-egy újabb gyermek folyamatot ezek
kiértékelésére. Ha nem sikerül új folyamatot indítani, a program áttér egy naiv rekurzív, de szekvenciális
megoldásra. Ha az első folyamat eredményéből látszik, hogy a feladat megoldható, akkor a gyermek nem várja
be a második folyamat eredményét.
\\
Mind a két gyermek bevárása után a részfeladat eredménye mindenképp ismert lesz. Ilyenkor a futó gyermekfolyamat
ezt az eredményt elküldi az eredményt a szülőnek, és terminál.

\subsection{Fordítás menete}

A fordítás a tantárgy honlapjáról letöltött \verb|Makefile.aimk| fájl és az \verb|aimk| eszköz
segítségével történik. A makefile tartalmából kiderül, hogy a kódban használhatjuk a C++11-es
szabványos elemeket.
A modulokat úgy jelöljük ki fordításra, hogy hozzáfűzzük a makefile első sorához a nevüket,
kiterjesztés nélkül.

\subsection{Tesztelés}

A program tesztelése két részből állt, az egyik a megoldás helyességét vizsgálta, a másik
a kód sebességét.

A helyesség vizsgálatához az alapeseteket teszteltük, illetve a beadandó leírásánál mellékelt
példabemenetek le lettek futtatva egy-egy helyes és helytelen összeggel.

A sebesség tesztelésére lefuttattam a legnagyobb elemszámú bemenetet az atlaszon különböző mennyiségű PVM-m blade
hozzáadásával. Az eredmények alább láthatóak.\\[12px]

	\begin{tabular}{l|*{14}{c}}
		Bladek száma & 0 & 1 & 2 & 4 \\
		\hline
		Teszt1 & 0.061 & 0.061 & 0.031 & 0.053 \\
		\hline
		Teszt2 & 0.061 & 0.036 & 0.033 & 0.279 \\
		\hline
		Teszt3 & 0.072 & 0.063 & 0.357 & 0.139 \\
	\end{tabular}
\\[12px]

Amint láthatjuk, a blade-ek száma ebben az esetben nem pozitívan, hanem inkább negatívan befolyásolja a program futásidejét.
Ez a jelenség két dolognak tulajdonítható. Az egyik az, hogy egy-egy gyermekfolyamat által végzett számítás elenyésző,
a folyamatok létrehozása és az üzenetek küldése több időt vesz igénybe. A másik az, olyan elemszámokra is párhuzamosan
történik a kiértékelés, amikre a szekvenciális kiértékelés gyorsabb lenne. Ez a két tényező együtt fokozottan rontja az
algoritmus teljesítményét. Megoldás lehetne egy adott elemszám alatt átváltani szekvenciális számításra, illetve egy-egy
gyermekfolyamatnak komplexebb alfeladatot adni.

\end{document}