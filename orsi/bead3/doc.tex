\documentclass[12pt]{article}

%margó méretek
\usepackage[a4paper,
inner = 25mm,
outer = 25mm,
top = 25mm,
bottom = 25mm]{geometry}


%packagek, ha használni akarunk valamit menet közben
\usepackage{lmodern}
\usepackage[magyar]{babel}
\usepackage[utf8]{inputenc}
\usepackage[T1]{fontenc}
\usepackage[hidelinks]{hyperref}
\usepackage{graphicx}
\usepackage{amssymb}
\usepackage{epstopdf}
\usepackage{setspace}
\usepackage[nottoc,numbib]{tocbibind}
\usepackage{setspace}
\usepackage{spverbatim}
\usepackage[normalem]{ulem}

\setstretch{1.2}
\begin{document}
	
	

% a címlap, túl sokat nem kell módosítani rajta, leszámítva a neved/neptunodat (dátumot)
\begin{titlepage}
	\vspace*{0cm}
	\centering
	\begin{tabular}{cp{1cm}c}
		\begin{minipage}{4cm}
			\vspace{0pt}
			\includegraphics[width=1\textwidth]{elte_cimer}
		\end{minipage} & &
		\begin{minipage}{7cm}
			\vspace{0pt}Eötvös Loránd Tudományegyetem \vspace{10pt} \newline
			Informatikai Kar \vspace{10pt} \newline
			Programozási Nyelvek és Fordítóprogramok Tanszék
		\end{minipage}
	\end{tabular}
	
	\vspace*{0.2cm}
	\rule{\textwidth}{1pt}
	
	\vspace*{3cm}
	{\Huge Osztott rendszerek specifikációja és implementációja }
	
	\vspace*{0.5cm}
	{\normalsize IP-08bORSIG}
	
	\vspace{2cm}
	{\huge Dokumentáció a 3. beadandóhoz}
	
	\vspace*{5cm}
	
	{\large \verb|Abonyi-Tóth Ádám| } % (név)
	
	{\large \verb|DKC31P| }  % (neptun)
		
	
	\vfill
	
	\vspace*{1cm}
	2017. december 29. % (dátum)
\end{titlepage}

\section{Kitűzött feladat}

Egy előre megadott fájlban képek szöveges reprezentációját találhatjuk (pixelenként RGB módon). \\
\\
A program parancssori paraméterként kapja meg az alábbiakat:
\begin{itemize}
	\item A képek átméretezési arányát \%-ban. (pl. 50 - ekkor 50\%-ra, azaz felére kell csökkenteni az összes képet. 25 esetén negyed-méretet kapnánk, etc.)
	\item Annak a fájlnak a neve, a képek primitív leírását tartalmazza. (pl. \\ \verb|'pictures.txt'|)
	\item A kimeneti fájl neve (pl. \verb|'picross.quiz'|)
\end{itemize}
Egy képen végrehajtandó transzformáció a következő lépések kompozíciójával áll elő:
\begin{itemize}
	\item A fájlból (első param.) beolvasott képeket először a megadott arányban át kell méretezni.
	\item Ezek után az így kapott kisebb képek színeit kell leképezni az előre megadott 8 szín valamelyikére.
	\item Ezt követően az így kapott ábrákban minden sorra és oszlopra ki kell számolni, hogy egymás után hány azonos színű pixelt láthatunk (de nem szimplán azt, hogy az adott sorban/oszlopban hány különböző szín található).
\end{itemize}

A kapott eredményeket (a méretezett és megfelelő színre konvertált képeket és a hozzájuk tartozó címkéket) írjuk ki a kimeneti fájlba (3-ik paraméter)!

\section{Felhasználói dokumentáció}

\subsection{Rendszer-követelmények, telepítés}

A program futtatásához szükséges a PVM rendszer megléte. Telepítéséhez a négy bináris
állomány, a \textit{master}, \textit{first}, \textit{second} és a \textit{third} megfelelő helyen való elhelyezése szükséges.

\subsection{A program használata}

A program futtatásához előbb lépjünk be PVM-be, és inicializáljuk tetszés szerint a blade-eket.
Az indítás a PVM-en belül a következő paranccsal történik:
\\
\verb|spawn -> master rate input.txt output.txt|,
\\
ahol \verb|rate| az átméretezés mértéke százalékban, \verb|input.txt| a bemeneti fájl neve, \verb|output.txt| pedig
a kimeneti fájl neve a felhasználó könyvtárához képest.
\\
Figyeljünk az inputfájlban található adatok helyességére és megfelelő tagolására, mivel
az alkalmazás külön ellenőrzést nem végez erre vonatkozóan. Futás után a paraméterben megadott
fájl tartalmazza a kapott eredményt.

\section{Fejlesztői dokumentáció}

\subsection{Megoldási mód}

A programot az \textit{adatcsatorna tételére} vezetjük vissza.
A kódot logikailag három egységre osztjuk:
\begin{itemize}
	\item \textit{Főfolyamat:} A szülőfolyamat feladata a bemenet és a kimenet kezelése és az adatcsatorna felállítása.
	\item \textit{Gyermekfolyamatok:} A 3 gyermek feladata az egyes függvénykomponensek ($ f_1 $, $ f_2 $ és $ f_3 $) végrehajtása.
			\sout{És hogy megvédjék a Földet az Angyalok támadásaitól és a Third Impact-től.}
	\item \textit{Model:} A \verb|pvm_utils.hpp|, \verb|image.hpp| és \verb|resize.hpp| fájlok szerepe a folyamatok közös típusainak és segédfüggvények különválasztása a tesztelhetőség és a karbantarthatóság érdekében.
\end{itemize}

\subsection{Implementáció}

A feladat alapvető típusai az \verb|Image| (RGB kép reprezentációja), a \verb|Image3bit| (3 bites kép ábrázolása),
és a \verb|ColorTag = std::vector<std::vector<int>>| (a címkék reprezentációja). \\
\\
A \verb|master.cpp| fájlban van a főprogram megvalósítása. A \verb|main| függvény beolvassa a bemeneti fájlból a képeket,
majd a PVM segítségével létrehoz 3 gyermeket, és elküldi nekik az adatcsatorna létrehozásához szükséges információkat.
Ezután amíg van adat, fogadja a megoldásokat a harmadik gyermektől. \\
\\
A gyermekek egy általános sémát követnek, amit a \verb|pvm_utils::ChannelChild| típusban rögzítünk. Ez az osztály biztosítja azt a működést,
hogy ha adott egy függvényt, akkor amíg a csatornán van adat, addig leveszi a bemenetet a csatornáról, alkalmazza rá a függvényt, és
továbbítja az eredményt a következő csatornán. Ezt az osztályt használva az egyes gyermekeknél elég megadni magát az $ f_i $ függvényt. \\
\\
Az \verb|first| feladata az átméretezés, bemenete az arány és a forráskép \\ (\verb|std::tuple<int,Image>|), kimenete pedig a lekicsinyített kép (\verb|Image|). A megvalósítás \textit{Divide \& Conquer} stratégiát használ, ahol a kép részeit (\verb|ImgPart|) vagy további négy részre osztjuk (\verb|slicing::slice|), elvégezzük új szálon a számításokat, és egyesítjük őket (\verb|slicing::combine|), vagy ha elég kicsik, akkor kiszámoljuk az eredeti kép alapján a megfelelő pixel értékeket. \\
\\
A \verb|second| feladata a színek (\verb|Color|) konvertálása 3 bites tartományba (\verb|Color3bit|), így a bemeneti \verb|Image|-et \textit{Task farm} stratégiával soronként átalakítva visszaadja az \verb|Image3bit| eredményt. \\
\\
A \verb|third| gyermek a hárombites képre előállítja a címkéket (\verb|ColorTag|). Ez soronként és oszloponként szintén \textit{Task farm} segítségével történik. Amíg a task farm részfeladatai számolnak, a főfolyamat visszaalakítja a hárombites képet \verb|Image| típusúra.

\subsection{Fordítás menete}

A fordítás a tantárgy honlapjáról letöltött \verb|Makefile.aimk| fájl és az \verb|aimk| eszköz
segítségével történik. A makefile tartalmából kiderül, hogy a kódban használhatjuk a C++11-es
szabványos elemeket.
A modulokat úgy jelöljük ki fordításra, hogy hozzáfűzzük a makefile első sorához a nevüket,
kiterjesztés nélkül.

\subsection{Tesztelés}

A program helyességének tesztelése szekvenciális módon történt, lokálisan. Ezt főként az tette lehetővé, hogy minden PVM-specifikus kód a \verb|pvm_utils| névtérben van elkülönítve, és független a többi egység működésétől, így PVM-et nélkülöző rendszerben is tesztelhető volt az alkalmazás. \\
\\
Ezt követte a kommunikáció tesztelése az Atlaszon, ahol a program szintén helyesen működött a példabemenetekre. \\
\\
Beláthatjuk, hogy a pvm rendszerben háromnál több blade használata már nem fogja tovább gyorsítani a futtatást, mivel pvm-es gyermek csak 3 van, 3 aktív blade-nél viszont kell ahhoz, hogy mindegyik gyermek a lehető legtöbb erőforráshoz jusson. 


\end{document}